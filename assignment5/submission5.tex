\documentclass{scrartcl}
\usepackage[utf8]{inputenc}
%\usepackage[T1]{fontenc}
\usepackage[a4paper, left=2.5cm, right=2.5cm, top=2.5cm, bottom=4cm]{geometry}
\usepackage[english]{babel}
\usepackage{amsmath, amsthm, amssymb, amstext}
\usepackage{listings}
\usepackage{color}
\usepackage{graphicx}
\usepackage{xparse}
\usepackage{fancyhdr}
\usepackage{algorithmicx}
\usepackage{algpseudocode}
\usepackage{algorithm}
\usepackage{parskip}
\usepackage[table]{xcolor}
\usepackage{tabularx}
\usepackage{enumerate}
\usepackage{enumitem}
\usepackage{float}
%\usepackage{minted}
\usepackage {tikz}
\usetikzlibrary{positioning}

\pagestyle{fancy}


\rhead{{\newcommand\and\\\getauthors}}
\author{Felix Bühler\\2973410 \and Clemens Lieb\\3130838 \and Steffen Wonner\\2862123 \and Fabian Bühler\\2953320}
\lhead{\textbf\gettitle}
\title{\gettitle}
\chead{\getsubtitle}
\subtitle{\getsubtitle}

\addtolength{\headheight}{2\baselineskip}
\renewcommand{\headrulewidth}{0pt}

\newcommand{\gettitle}{Distributed systems I\\Winter Term 2019/20}
\newcommand{\getsubtitle}{G2T1 – Assignment 5 (theoretical part)}
\newcommand{\getauthors}{Felix Bühler \and Clemens Lieb \and Steffen Wonner \and Fabian Bühler}
\setlength{\headheight}{53pt}

\begin{document}
\maketitle

\section*{1 - Two-Phase Locking}
\subsection*{a)}

See figure \ref{fig:1a}

\begin{figure}[ht!]
\label{fig:1a}
\begin{tikzpicture}
    \draw[->] (0, 15) -- (0, 0);
    \draw[->] (3, 15) -- (3, 0);
    \draw[->] (6, 15) -- (6, 0);
    \draw[->] (9, 15) -- (9, 0);

    \foreach \t/\p in {1/0,2/3,3/6,4/9} {
      \draw node at (\p, 16) {\(T_\t\)};
    };

    % transaction 4 can just be parallel
    \node at ( 8, 14.5) {\(rL_4[u]\)};
    \node at ( 8, 13.5) {\(r_4[u]\)};
    \node at ( 8, 12.5) {\(rL_4[o]\)};
    \node at ( 8, 11.5) {\(r_4[o]\)};
    \node at ( 8, 10.5) {\(wL_4[y]\)};
    \node at ( 8,  9.5) {\(rU_4[u]\)};
    \node at ( 8,  8.5) {\(rU_4[o]\)};
    \node at ( 8,  7.5) {\(w_4[y]\)};
    \node at ( 8,  6.5) {\(wU_4[y]\)};
    \node at ( 8,  5.5) {\(c_4\)};

    % transaction 1 does not need to wait
    \node at (-1, 14.5) {\(rL_1[v]\)};
    \node at (-1, 13.5) {\(r_1[v]\)};
    \node at (-1, 12.5) {\(wL_1[v]\)};
    \node at (-1, 11.5) {\(w_1[v]\)};
    \node at (-1, 10.5) {\(wL_1[x]\)};
    \node at (-1,  9.5) {\(rU_1[v]\)};
    \node at (-1,  8.5) {\(wU_1[v]\)};
    \node at (-1,  7.5) {\(w_1[x]\)};
    \node at (-1,  6.5) {\(wU_1[x]\)};
    \node at (-1,  5.5) {\(a_1\)};

    %transaction 2 needs to wait for wU[x]
    \node at (2, 14.5) {\(wL_2[z]\)};
    \node at (2, 13.5) {\(w_2[z]\)};
    \node at (2,  5.5) {\(rL_2[x]\)};
    \node at (2,  4.5) {\(wU_2[z]\)};
    \node at (2,  3.5) {\(r_2[x]\)};
    \node at (2,  2.5) {\(rU_2[x]\)};
    \node at (2,  1.5) {\(c_2\)};

    %transaction 3 needs to wait for wU[v]
    \node at (5, 14.5) {\(rL_3[s]\)};
    \node at (5, 13.5) {\(r_3[s]\)};
    \node at (5,  8.5) {\(rL_3[v]\)};
    \node at (5,  7.5) {\(r_3[v]\)};
    \node at (5,  6.5) {\(wL_3[s]\)};
    \node at (5,  5.5) {\(rU_3[v]\)};
    \node at (5,  4.5) {\(rU_3[s]\)};
    \node at (5,  3.5) {\(w_3[s]\)};
    \node at (5,  2.5) {\(wU_3[s]\)};
    \node at (5,  1.5) {\(c_3\)};

    % draw the event markings in the timelines
    \foreach \x/\y in {0/14.5,0/13.5,0/12.5,0/11.5,0/10.5,0/9.5,0/8.5,0/7.5,0/6.5,0/5.5,%
                       3/14.5,3/13.5,3/5.5,3/4.5,3/3.5,3/2.5,3/1.5,%
                       6/14.5,6/13.5,6/8.5,6/7.5,6/6.5,6/5.5,6/4.5,6/3.5,6/2.5,6/1.5,%
                       9/14.5,9/13.5,9/12.5,9/11.5,9/10.5,9/9.5,9/8.5,9/7.5,9/6.5,9/5.5} {
        \draw[-] (\x-.25,\y) -- (\x+.25,\y);
    };
\end{tikzpicture}
\caption{Operation timeline for History \(H_1\) using non-strict two-phase locking}
\end{figure}

\subsection*{b)}

The transactions 2 and 3 need to be aborted / rolled back.
This is because the abort of transaction 1 does not happen until after the locks for reading uncommitted data have been acquired, resulting in incorrect reads.
To enforce consistency, the transactions must be aborted when transaction 1 is aborted.

\subsection*{c)}

See figure \ref{fig:1c}

\begin{figure}[ht!]
    \label{fig:1c}
    \begin{tikzpicture}
        \draw[->] (0, 15) -- (0, -2);
        \draw[->] (3, 15) -- (3, -2);
        \draw[->] (6, 15) -- (6, -2);
        \draw[->] (9, 15) -- (9, -2);
    
        \foreach \t/\p in {1/0,2/3,3/6,4/9} {
          \draw node at (\p, 16) {\(T_\t\)};
        };
    
        % transaction 4 can just be parallel
        \node at ( 8, 14.5) {\(rL_4[u]\)};
        \node at ( 8, 13.5) {\(r_4[u]\)};
        \node at ( 8, 12.5) {\(rL_4[o]\)};
        \node at ( 8, 11.5) {\(r_4[o]\)};
        \node at ( 8, 10.5) {\(wL_4[y]\)};
        \node at ( 8,  9.5) {\(w_4[y]\)};
        \node at ( 8,  8.5) {\(c_4\)};
        \node at ( 8,  7.5) {\(rU_4[u]\)};
        \node at ( 8,  6.5) {\(rU_4[o]\)};
        \node at ( 8,  5.5) {\(wU_4[y]\)};
    
        % transaction 1 does not need to wait
        \node at (-1, 14.5) {\(rL_1[v]\)};
        \node at (-1, 13.5) {\(r_1[v]\)};
        \node at (-1, 12.5) {\(wL_1[v]\)};
        \node at (-1, 11.5) {\(w_1[v]\)};
        \node at (-1, 10.5) {\(wL_1[x]\)};
        \node at (-1,  9.5) {\(w_1[x]\)};
        \node at (-1,  8.5) {\(a_1\)};
        \node at (-1,  7.5) {\(rU_1[v]\)};
        \node at (-1,  6.5) {\(wU_1[v]\)};
        \node at (-1,  5.5) {\(wU_1[x]\)};
    
        %transaction 2 needs to wait for wU[x]
        \node at (2, 14.5) {\(wL_2[z]\)};
        \node at (2, 13.5) {\(w_2[z]\)};
        \node at (2,  4.5) {\(rL_2[x]\)};
        \node at (2,  3.5) {\(r_2[x]\)};
        \node at (2,  2.5) {\(c_2\)};
        \node at (2,  1.5) {\(wU_2[z]\)};
        \node at (2,  0.5) {\(rU_2[x]\)};
    
        %transaction 3 needs to wait for wU[v]
        \node at (5, 14.5) {\(rL_3[s]\)};
        \node at (5, 13.5) {\(r_3[s]\)};
        \node at (5,  5.5) {\(rL_3[v]\)};
        \node at (5,  4.5) {\(r_3[v]\)};
        \node at (5,  3.5) {\(wL_3[s]\)};
        \node at (5,  2.5) {\(w_3[s]\)};
        \node at (5,  1.5) {\(c_3\)};
        \node at (5,  0.5) {\(rU_3[s]\)};
        \node at (5,  -0.5) {\(rU_3[v]\)};
        \node at (5,  -1.5) {\(wU_3[s]\)};
    
        % draw the event markings in the timelines
        \foreach \x/\y in {0/14.5,0/13.5,0/12.5,0/11.5,0/10.5,0/9.5,0/8.5,0/7.5,0/6.5,0/5.5,%
                           3/14.5,3/13.5,3/4.5,3/3.5,3/2.5,3/1.5,3/0.5,%
                           6/14.5,6/13.5,6/5.5,6/4.5,6/3.5,6/2.5,6/1.5,6/0.5,6/-0.5,6/-1.5%
                           9/14.5,9/13.5,9/12.5,9/11.5,9/10.5,9/9.5,9/8.5,9/7.5,9/6.5,9/5.5} {
            \draw[-] (\x-.25,\y) -- (\x+.25,\y);
        };
    \end{tikzpicture}
    \caption{Operation timeline for History \(H_1\) using strict two-phase locking}
    \end{figure}

\section*{2 - Two-Phase Commit}
\subsection*{a)}
\subsection*{b)}
\subsection*{c)}

\section*{3 - Data Replication}
\subsection*{a)}
\subsection*{b)}
\subsection*{c)}

\end{document}
