\documentclass{scrartcl}
\usepackage[utf8]{inputenc}
%\usepackage[T1]{fontenc}
\usepackage[a4paper, left=2.5cm, right=2.5cm, top=2.5cm, bottom=4cm]{geometry}
\usepackage[ngerman]{babel}
\usepackage{amsmath, amsthm, amssymb, amstext}
\usepackage{listings}
\usepackage{color}
\usepackage{graphicx}
\usepackage{xparse}
\usepackage{fancyhdr}
\usepackage{algorithmicx}
\usepackage{algpseudocode}
\usepackage{algorithm}
\usepackage{parskip}
\usepackage[table]{xcolor}
\usepackage{tabularx}
\usepackage{enumerate}
\usepackage{enumitem}
%\usepackage{minted}
\usepackage {tikz}
\usetikzlibrary {positioning}

\pagestyle{fancy}


\rhead{{\newcommand\and\\\getauthors}}
\author{Felix Bühler\\2973410 \and Clemens Lieb\\3130838 \and Steffen Wonner\\2862123 \and Fabian Bühler\\2953320}
\lhead{\textbf\gettitle}
\title{\gettitle}
\chead{\getsubtitle}
\subtitle{\getsubtitle}

\addtolength{\headheight}{2\baselineskip}
\renewcommand{\headrulewidth}{0pt}

\newcommand{\gettitle}{Distributed systems I\\Winter Term 2019/20}
\newcommand{\getsubtitle}{G2T1 – Assignment 2 (theoretical part)}
\newcommand{\getauthors}{Felix Bühler \and Clemens Lieb \and Steffen Wonner \and Fabian Bühler}
\setlength{\headheight}{53pt}

\begin{document}
\maketitle

\section*{1 - Parameter Passing and RMI}
\subsection*{a)}
\begin{enumerate}[label=(\roman*)]
	% no return value, so a is never changed
	\item \texttt{[1, 2, 3, 4, 5]}
	% the code given does not perform a swap, it only overwrites the lower indices with values from the upper ones.
	% because of reference semantics that overwrites the second input as well
	\item \texttt{[5, 4, 3, 4, 5]}
	\item \texttt{[5, 4, 3, 2, 1]} \emph{or} \texttt{[1, 2, 3, 4, 5]} \emph{or} \texttt{[5, 4, 3, 4, 5]}, depending on the exact semantic behaviour of the copy and restore operations.
	Given that the remote server passes the arguments by reference, the first question is whether \textit{x} and \textit{y} have separate copies of \textit{a}. 
	If they do, the first or second option is possible. 
	If there are no separate copies of \textit{a}, the procedure behaves as if the parameters were passed by reference. 
	Otherwise it depends on which value is restored first.
\end{enumerate}

\subsection*{b)}

The result printed to the system console is:
\begin{verbatim}
$ java StringServer
1: B
2: BRS
3: ABSBRS
\end{verbatim}

We can deduce that all objects that extend UnicastRemoteObject are sent as reference, all other objects are sent as value.
In the first and last two lines the parameter is a StringBuffer object which does not implement UnicastRemoteObject and is therefore sent as value. 
In the third line the parameter is a StringServer object which does implement UnicastRemoteObject and because of this is sent as reference.

\subsection*{c)}

The first line shows that even "complex java objects" are passed by value so long as they are not remote objects. 
This implies that even very large non-remote objects would be serialized and passed by value (e.g. Arrays with hundreds of thousands of elements).
Such a behaviour incurs large overheads, especially if only parts of the passed value are accessed by the remote procedure.

The lines two and three demonstrate the reference-passing behaviour within java's RMI implementation. 
Notably, remote objects are passed by reference. 
Calls to the passed remote objects are routed back to the JVM the actual object resides in. 
This can lead to large overheads if a lot of calls are made to the referenced object.

\section*{2 - Chord System}
\subsection*{a)}

	\begin{minipage}[t]{.25\linewidth}
        \centering
        Node 4:\\
			\begin{tabular}{|l|l|}
				\hline
				1 & 11 \\ \hline
				2 & 11 \\ \hline
				3 & 11 \\ \hline
				4 & 16 \\ \hline
				5 & 23 \\ \hline
			\end{tabular}
	\end{minipage}%
	\begin{minipage}[t]{.25\linewidth}
        \centering
        Node 11:\\
		\begin{tabular}{|l|l|}
			\hline
			1 & 16 \\ \hline
			2 & 16 \\ \hline
			3 & 16 \\ \hline
			4 & 19 \\ \hline
			5 & 31 \\ \hline
		\end{tabular}
	\end{minipage}%
	\begin{minipage}[t]{.25\linewidth}
        \centering
        Node 16:\\
		\begin{tabular}{|l|l|}
			\hline
			1 & 19 \\ \hline
			2 & 19 \\ \hline
			3 & 23 \\ \hline
			4 & 26 \\ \hline
			5 & 4 \\ \hline
		\end{tabular}
    \end{minipage}
    \begin{minipage}[t]{.25\linewidth}
        \centering
        Node 19:\\
        \begin{tabular}{|l|l|}
            \hline
            1 & 23 \\ \hline
            2 & 23 \\ \hline
            3 & 23 \\ \hline
            4 & 31 \\ \hline
            5 & 4 \\ \hline
        \end{tabular}
    \end{minipage}%

	\begin{minipage}[t]{.25\linewidth}
		\centering
        Node 23:\\
		\begin{tabular}{|l|l|}
			\hline
			1 & 26 \\ \hline
			2 & 26 \\ \hline
			3 & 31 \\ \hline
			4 & 31 \\ \hline
			5 & 11 \\ \hline
		\end{tabular}
	\end{minipage}%
	\begin{minipage}[t]{.25\linewidth}
		\centering
        Node 26:\\
		\begin{tabular}{|l|l|}
			\hline
			1 & 31 \\ \hline
			2 & 31 \\ \hline
			3 & 31 \\ \hline
			4 & 4 \\ \hline
			5 & 11 \\ \hline
		\end{tabular}
	\end{minipage}
	\begin{minipage}[t]{.25\linewidth}
		\centering
        Node 31:\\
		\begin{tabular}{|l|l|}
			\hline
			1 & 4 \\ \hline
			2 & 4 \\ \hline
			3 & 4 \\ \hline
			4 & 11 \\ \hline
			5 & 16 \\ \hline
		\end{tabular}
	\end{minipage}


\subsection*{b)}
\begin{enumerate}
	\item node 4 $ \rightarrow $ node 16
	\item node 16 $ \rightarrow $ node 19
	\item node 19 $ \rightarrow $ node 23
\end{enumerate}

\subsection*{c)}
\begin{minipage}[t]{.25\linewidth}
	\centering
    Node 24:\\
	\begin{tabular}{|l|l|}
		\hline
		1 & 26 \\ \hline
		2 & 26 \\ \hline
		3 & 31 \\ \hline
		4 & 4 \\ \hline
		5 & 11 \\ \hline
	\end{tabular}
\end{minipage}

The finger table of node 16 will change eventually while the one for node 23 will change immediately.

\subsection*{d)}
If both nodes join simultaneously they are between the same two nodes (4 and 11). 
If they now try to each update their predecessor and successor then we have a race condition while updating a linked list. 
This can leave the list (or in this case ring) in an inconsistent state that makes routing impossible!

\subsection*{e)}

This code updates the FT entries of \textit{p} for the former successor's predecessor which has the id \textit{q}.

\begin{algorithm}
\begin{algorithmic}
	\For{$i \gets 0 \textbf{ to } \lfloor log_2(q - p) \rfloor$}
	   \State $FT_p[i] \gets q$
	\EndFor
\end{algorithmic}
\end{algorithm}

This code assumes that only \textit{q} has been inserted between \textit{p} and the former successor.
The obvious issue arising from that (namely that multiple insertions are not correctly handled), can be easily remedied by repeating the consistency check after updating the finger table.
As presented the code is already eventually consistent, though.

\subsection*{f)}

\section*{3 - Name Services}

\subsection*{a)}
\begin{center}
	\begin{tabular}{|c|c|c|}
		\hline
		& Iterative resolution & Recursive resolution \\ \hline
		1. Lookup & 10 & 10 \\ \hline
		2. Lookup & 10 & 10 \\ \hline
		3. Lookup & 8 & 8 \\ \hline
	\end{tabular}
\end{center}

\subsection*{b)}
\begin{center}
	\begin{tabular}{|c|c|c|}
		\hline
		& Iterative resolution & Recursive resolution \\ \hline
		1. Lookup & 10 & 10 (8 fast, 2 slow) \\ \hline
		2. Lookup & 10 & 8 (6 fast, 2 slow) \\ \hline
		3. Lookup & 8 & 8 (6 fast, 2 slow) \\ \hline
	\end{tabular}
\end{center}

\subsection*{c)}

\lstset{basicstyle=\ttfamily\footnotesize,breaklines=true}
\subsubsection*{i.}
\begin{lstlisting}
$ dig NS uni-stuttgart.de
...
;; ANSWER SECTION:
uni-stuttgart.de.	2191	IN	NS	minnehaha.rhrk.uni-kl.de.
uni-stuttgart.de.	2191	IN	NS	dns3.belwue.de.
uni-stuttgart.de.	2191	IN	NS	dns1.belwue.de.
uni-stuttgart.de.	2191	IN	NS	dns0.uni-stuttgart.de.
uni-stuttgart.de.	2191	IN	NS	dns1.uni-stuttgart.edu.
...
\end{lstlisting}
There are 5 server responsible for the DNS entry of 'uni-stuttgart.de':
\begin{itemize}
	\item minnehaha.rhrk.uni-kl.de
	\item dns3.belwue.de
	\item dns1.belwue.de
	\item dns0.uni-stuttgart.de
	\item dns1.uni-stuttgart.edu
\end{itemize}
\subsubsection*{ii.}
\begin{lstlisting}
$ dig -x 129.69.216.249
...
;; ANSWER SECTION:
249.216.69.129.in-addr.arpa. 86397 IN	PTR	ipvslogin.informatik.uni-stuttgart.de.
...
\end{lstlisting}
PTR is usefull for resolving an IP address to a domain or hostname (reverse DNS lookup).\\
The given IP resolves to: 'ipvslogin.informatik.uni-stuttgart.de'
\subsubsection*{iii.}
\begin{lstlisting}
$ dig +trace www.uni-stuttgart.de
...
.			347153	IN	NS	h.root-servers.net.
.			347153	IN	NS	l.root-servers.net.
.			347153	IN	NS	d.root-servers.net.
.			347153	IN	NS	m.root-servers.net.
.			347153	IN	NS	e.root-servers.net.
.			347153	IN	NS	k.root-servers.net.
.			347153	IN	NS	i.root-servers.net.
.			347153	IN	NS	b.root-servers.net.
.			347153	IN	NS	c.root-servers.net.
.			347153	IN	NS	a.root-servers.net.
.			347153	IN	NS	f.root-servers.net.
.			347153	IN	NS	g.root-servers.net.
.			347153	IN	NS	j.root-servers.net.
...
;; Received 553 bytes from 192.168.0.1#53(192.168.0.1) in 28 ms

de.			172800	IN	NS	z.nic.de.
de.			172800	IN	NS	f.nic.de.
de.			172800	IN	NS	s.de.net.
de.			172800	IN	NS	a.nic.de.
de.			172800	IN	NS	l.de.net.
de.			172800	IN	NS	n.de.net.
...
;; Received 802 bytes from 2001:500:12::d0d#53(g.root-servers.net) in 22 ms

uni-stuttgart.de.	86400	IN	NS	dns0.uni-stuttgart.de.
uni-stuttgart.de.	86400	IN	NS	dns1.belwue.de.
uni-stuttgart.de.	86400	IN	NS	dns1.uni-stuttgart.edu.
uni-stuttgart.de.	86400	IN	NS	dns3.belwue.de.
uni-stuttgart.de.	86400	IN	NS	minnehaha.rhrk.uni-kl.de.
...
;; Received 619 bytes from 2001:67c:1011:1::53#53(n.de.net) in 25 ms

www.uni-stuttgart.de.	3600	IN	A	129.69.8.19
...
uni-stuttgart.de.	3600	IN	NS	dns0.uni-stuttgart.de.
uni-stuttgart.de.	3600	IN	NS	minnehaha.rhrk.uni-kl.de.
uni-stuttgart.de.	3600	IN	NS	dns1.uni-stuttgart.edu.
uni-stuttgart.de.	3600	IN	NS	dns3.belwue.de.
uni-stuttgart.de.	3600	IN	NS	dns1.belwue.de.
...
;; Received 1365 bytes from 129.143.2.10#53(dns1.belwue.de) in 25 ms

\end{lstlisting}
The path the lookup took was:
\begin{enumerate}
	\item 192.168.0.1
	\item g.root-servers.net
	\item n.de.net
	\item dns1.belwue.de
\end{enumerate}

\end{document}
