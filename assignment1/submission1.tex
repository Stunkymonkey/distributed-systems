\documentclass{scrartcl}
\usepackage[utf8]{inputenc}
%\usepackage[T1]{fontenc}
\usepackage[a4paper, left=2.5cm, right=2.5cm, top=2.5cm, bottom=2.5cm]{geometry}
\usepackage[ngerman]{babel}
\usepackage{amsmath, amsthm, amssymb, amstext}
\usepackage{listings}
\usepackage{color}
\usepackage{graphicx}
\usepackage{xparse}
\usepackage{fancyhdr}
\usepackage{algorithmicx}
\usepackage{parskip}
\usepackage[table]{xcolor}
\usepackage{tabularx}
\usepackage{enumerate}
%\usepackage{minted}
\usepackage {tikz}
\usetikzlibrary {positioning}

\pagestyle{fancy}


\rhead{{\newcommand\and\\\getauthors}}
\author{Felix Bühler\\2973410 \and Clemens Lieb\\3130838 \and Steffen Wonner\\2862123 \and Fabian Bühler\\2953320}
\lhead{\textbf\gettitle}
\title{\gettitle}
\chead{\getsubtitle}
\subtitle{\getsubtitle}

\addtolength{\headheight}{2\baselineskip}
\renewcommand{\headrulewidth}{0pt}

\newcommand{\gettitle}{Verteilte Systeme I\\Winter Term 2019/20}
\newcommand{\getsubtitle}{G2T1 – Assignment 1 (theoretical part)}
\newcommand{\getauthors}{Felix Bühler \and Clemens Lieb \and Steffen Wonner \and Fabian Bühler}
\setlength{\headheight}{53pt}

\begin{document}
\maketitle

\section*{1 Transparency Levels}
\subsection*{a)}
The access is location transparent, because the location is hidden behind an unresolved URL.
The actual adress of the webservice is automatically resolved through DNS acting as service discovery mechanism.
\subsection*{b)}
The service is not replication transparent because there is a unique name for every location that has to be checked separatly.
\subsection*{c)}
The access is replication transparent, because it is not apparent whether Otto checks one or multiple web services.
It's furthermore unknown which of the two replication locations did provide the file in the end.
\subsection*{d)}
\subsubsection*{i.}
It is only possible to write all copies at once so all copies are always the same.
This iplies that independently from when or which copy is read the answer is always up to date.
\subsubsection*{ii.}
Write operations are performed on \(n-1\) copies, as required by the locking constraints.
Read operations are performed on \(2\) different randomly selected copies, the client accepts the "latest available" version.

\section*{2 System Models}
\subsection*{a)}
\begin{itemize}
	\item No message is lost
	\item The maximum possible delay is known
	\item There is a limited amount of clock skew between the processes
\end{itemize}
\subsection*{b)}
\subsubsection*{i.}
Assuming a limited clock skew of \(\delta c\) between the processes, the process \(P_i\) sends a broadcast message \(2 * \delta c + \delta t\) time units after receiving \(i - 1\) messages.
This allows for instant delivery to a process skewed by the maximum amount of time ''forwards'' while keeping the order of messages intact when sending the message even with the maximum delay \(\delta t\) to a process skewed by the maximum amount of time ''backwards''.
\subsubsection*{ii.}
In the worst case, the synchronization protocol takes \(2n (\delta t + \delta c) + 2\delta c\) time units.
This assumes that all messages take the maximum possible amount of time to deliver, namely \(n delta t\) overall, with clock skew cancelling itself out across all processes.
Adding the wait-time for each process \(n(\delta t + 2\delta c)\) yields \(2n (\delta t + \delta c)\).
The final component is assuming the maximum clock skew between \(P_1 \text{ and } P_n\).

\section*{3 Three-Army-Problem}
\subsection*{a)}
\subsection*{b)}

\section*{4 System Availability}

\begin{enumerate}[a)]
	\item $ A_x = \dfrac{80t}{100t} = 80\% $

		$ A_y = \dfrac{60t}{100t} = 60\%$

	\item Overall statistical availability:

		$ A_S = 1 - P(Y=\text{down} \cup X=\text{down}) = 1 - ((1 - A_x) \cdot (1 - A_y)) = 1 - (0.2 \cdot 0.4) = 1 - 0.08 = 0.92 = 92\%$

		Availability in the concrete example:

		$ A_S = \dfrac{80t}{100t} = 80\% $

	\item $ P(Y=\text{up}|X=\text{up}) = \frac{P(Y=\text{up} \cap X=\text{up})}{P(X=\text{up})} = \frac{0.6}{0.8} = 75\% $

		$ P(Y=\text{up}|X=\text{down}) = \frac{P(Y=\text{up} \cap X=\text{down})}{P(X=\text{down})} = \frac{0}{0.2} = 0\% $

		Node $Y$ seems to depend on Node $X$ to be available.

	\item The availability of the system ($ A_S $) depends on the node $X$.

		$ A_S = P(Y=\text{up} \cup X=\text{up}) = P(X=\text{up}) = A_X = 0.8 $

		This is different from the result in b) ($ A_S = 0.92 $) where the assumption was that the nodes have independent failures.
		If the nodes are dependent then the overall system can not achieve a higher availability than any one of the nodes.

\end{enumerate}

\end{document}
